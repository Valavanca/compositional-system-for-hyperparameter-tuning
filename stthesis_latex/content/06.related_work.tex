\chapter{Related work}
    Many existing approaches can be categorized as multi-objective optimization. That is why
    introduce comparison criteria for a clear and concise demarcation of the approach presented in this thesis:

    Comparison Criteria for Related Work. 
    \begin{itemize}
        \item Variability. Exchange surrogate, solver and sampling algorithms as components. Variants on each optimization workflow step.
        \item Scalability. Extend single-objective problem on the fly to multi-objective.
        \item Adaptation. Surrogate portfolio.
        \item From 0 to hero. Sampling plan depends on surrogate validity. The Sobol sequence (and Latin hypercube).                
    \end{itemize}
    Important Features: Categorical variables, prior knowledge, multi-objective, feasibility constraints.


    % ------------------------------------------------------------------------------------------------------------------------------------------------------------------------
    % -------------------------------------------------------------------------------------------------------------------------------       Related work      ----------------
    % ------------------------------------------------------------------------------------------------------------------------------------------------------------------------

    \section{Dependencies}

    \paragraph{AUTO-SKLEARN} \cite{autosklearn:feurer2015efficient}
    - CASH (Combined Algorithm Selection and Hyperparameter optimization) problem

    \paragraph{TPOT} \cite{OlsonGECCO2016}
    Already implemented TPOT automodel as hypothesis candidate

    \paragraph{jMetalpy} \cite{benitezhidalgo2019jmetalpy}
    Partially implemented some solvers.

    \paragraph{Hyperopt}

    \paragraph{PlatEMO} \cite{PlatEMO}
    PlatEMO: A MATLAB Platform for Evolutionary Multi-Objective Optimization

    \paragraph{mlrMBO: A Modular Framework for Model-Based Optimization of Expensive Black-Box Functions}

    \paragraph{ParEGO}
    ParEGO was enhanced to a multi-point proposal by increasing the number of weight vectors randomly drawn in each iteration. If N points are desired, cN (c > 1) weight vectors are selected. 

\chapter{Foundation}
    Describe general objectives and there constraints

    % ----------------------------------- Parameter tuning
    \section{Parameter tuning}
        Define an objective function: 
        \begin{itemize}
            \item Accuracy
            \item Runtime
            \item Latency
            \item Energy
        \end{itemize}

        Optimization procedures:
        \begin{itemize}
            \item Grid search
            \item Random search
            \item Heuristics
        \end{itemize}
        Bayesian methods differ from random or grid search in that they use past evaluation results to choose the next values to evaluate.
        Limit expensive evaluations of the objective function by choosing the next input values based on those that have done well in the past.

        Optimization cost:
        \begin{itemize}
            \item Evaluation may be very expensive
            \item Sampling budget
            \item Possibly noisy
            \item Feasibility constraints
            \item Multi-objective
        \end{itemize}
        Ideally, we want a method that can explore the search space while also limiting evaluations of poor hyperparameter choices.

    
    \section{Multi-objective optimization}
        "Multi-objective optimization deals with such conflicting objectives. It provides a
        mathematical framework to arrive at optimal design state which accommodates the various criteria demanded by
        the application. The process of optimizing systematically and simultaneously a collection of objective functions
        are called multi-objective optimization (MOO)\cite{odugod2013}".
        Thus, a central issue of MOO is the relationship between the objective functions. These are usually modeled as preferences of the decision maker.

        Search or design space(Input space) -> Objective space(Output space).
        Practical optimization problems usually involve simultaneous optimization of multiple conflicting objectives with many constraints(?).

        Knowledge of the Pareto front enables the decision maker to visualize the consequences of his/her choices in terms of performance 
        for a criterion at the expense of one or other criteria, and to make appropriate decisions. Formally, a feasible vector x is said to (Pareto)-dominate another feasible vector x? if
        x is at least as good as x? for all the objectives, and strictly better than x? for at least one objective. The decision vectors in the feasible set 
        that are not dominated by any other feasible vector are called Pareto optimal. The set of non-dominated points in the 
        feasible is the set of Pareto solutions, whose images (by the objective functions) constitute the Pareto front.\cite{Audet2018PerformanceII}


        \subsection{Scalarizing. Weighted sum methods}
            Intuitively, Multi-Objective Optimization (MOO) could be
            facilitated by forming an alternative problem with a single,
            composite objective function using weighted sum. Single
            objective optimization techniques are then applied to this
            composite function to obtain a single optimal solution.
            However, the weighted sum methods have difficulties in
            selecting proper weight factors especially when there is no
            articulated a priori preference among objectives. Indeed, a
            posteriori preference articulation is usually preferred, because it
            allows a greater degree of separation between the optimization
            methodology and the decision-making process which also
            enables the algorithmic development process to be conducted
            independently of the application (Giagkiozos et al, 2015).
            Furthermore, instead of a single optimum produced by the
            weighted sum approach, MOO can yield a set of solutions
            exhibiting explicitly the tradeoff between different objectives. \cite{DBLP:journals/corr/abs-1812-07958}


            Why is the Weighting Method Ineffective?[Hirotaka Nakayama]
            Namely, it can not provide a solution among sunken parts of Pareto surface due to “duality gap” for not convex cases. 
            Even for convex cases, for example, in linear cases, even if we want to get a point in the middle of line segment between two vertices, we merely get a vertex of Pareto surface, as
            long as the well known simplex method is used. This implies that depending on the structure of problem, the linearly weighted sum can not necessarily provide a solution as DM desires.


        \subsection{Multy-Objective Evolutionary Algorithms}
            Multi-objective Evolutionary Algorithms (MOEAs) are common tools to solve optimization problems, 
            because of their applicability to complex fitness landscapes and solid performance on problems with large design spaces. 
            While other methods also exist, in this thesis we will focus on approaches using Evolutionary Algorithms for the Multy-objective optimizations.

            However, MOEAs still need many evaluations of the "black box" system to solve a typical real-world problem. 
            This is further complicated by the fact that many such problems are very expensive. Consolidated, this makes MOEAs unfeasible for costly and Multy-objective problem.
            A good solution is the integration of the surrogate model which extrapolate and approximate the fitness landscape from samples. MOEA use this surrogate model 
            as a target for optimization. Assumed that solution from surrogate close to a real solution.
            
            We want to understand if the performance of MOEAs approach can be improved by using compositional surrogates. 
            The key idea of compositional surrogates is the splitting objective space to multiple surrogates that extrapolate it independently. 
            Combination of multiple hypotheses should give them the potential to approximate more complicated problems. 

            The various surrogates are analysed on problems of differing complexity, from simple unimodal problems to problems with difficult multimodal. 

            \begin{itemize}
                \item Quality and Effort tradeoff for multi-objective
                \item Human in the loop: Composition technic as tools for domain expert
            \end{itemize}


        \subsection{Metrics for multy-objective solution}
            In single-objective minimization, the quality of a given solution is trivial to quantify:
            the smaller the objective function value, the better. However, evaluating the quality of an approximation of a Pareto set is non trivial.
            The question is important for the comparison of algorithms or prediction next configuration. According

            According to \cite{ZitzlerDT00}, a Pareto front approximation should satisfy the following:
            \begin{itemize}
                \item The distance between the Pareto front and its approximation should be minimized.
                \item A good (according to some metric) distribution of the points of the approximated front is desirable.
                \item The extent of the approximated front should be maximized, i.e., for each objective, a wide range of values should be covered by the non-dominated points.
            \end{itemize}

            The goal of metric is to answer this questions.  

            Metrics for performance indicators partitioned into four groups according to their properties \cite{Audet2018PerformanceII}: 
            \begin{itemize}
                \item cardinality
                \item convergence
                \item distribution
                \item spread
            \end{itemize}
        
            Keep making algorithmic progress toward the Pareto front in the objective function space.
            Straightforward applying of the coefficient of determination (R2) is the wrong indicator of success. 
        
            Evaluations of different sets of Pareto optimal points is multi-objective task.
            Objectives for improving pareto optimal solutions:
            - Keep hypervolume low. Reference point is 0 for all objectives
            - Maximize sparsity of points. Average distance. Crowding Distance
            - Percentage of non-dominant decisions in the total population
        
            Distribution and spread indicators According to \cite{CustodioMVV11}, “the spread metrics try to measure the extents of the spread achieved in a computed Pareto front approximation”. They are not really useful to evaluate the convergence of an algorithm, or at comparing algorithms. They only make sense when the Pareto set is composed of several solutions.
        
            For multi-objective optimization (MOO), an algorithm
            should provide a set of solutions that realize the optimal trade-offs between the considered optimization objectives, 
            i.e., Pareto set. Therefore, the performance comparison of MOO algorithms is based on their Pareto sets.
            In this study, two popular metrics Spacing metric(SM) and Hypervolume (HV) are used to quantify the performance of the algorithms. \cite{DBLP:journals/corr/abs-1812-07958}
           

            \paragraph{Pareto front evaluation}
            \begin{itemize}
                \item Hypervolume (HV)\cite{Zitzler2000ComparisonOM}. 
                This metric represents the volume of the objective space
                that is covered by the individuals of a non-dominated
                solutions set (solutions that belong to a Pareto front). The
                volume is delimited by two points: one point that is called
                the anti-optimal point (A) that is defined as the worst
                solution inside the objective space, and a second optimal
                point (pseudo-optimal) that is calculated by the proposed
                solution method. 
                \item Hyper-area Ratio (HR).
                The Hyper-area Ratio (HR) [24] employs the hypervolume of a solution set A
                divided by the hypervolume value of a Reference Front B. Higher values are
                preferred to lower ones.
                \item Pareto Dominance Indicator (ER). 
                Considers the solutions intersection between two given sets A and B, which can be 
                provided by different algorithms or used to compare a solution set S with a Pareto Front P.
                \item Crowding Distance. *pygmo2. The crowding distance value of a solution provides an estimate of the density of solutions set.
                \item Spacing \cite{Schott1995FaultTD}. Base on this metrics we can say what distribution of Pareto points. Less space metrics means better coverage of population target objectives.
                
            \end{itemize}
        
            Variants in evaluation of sets of solutions for each hypothesis.
            Each hypothesis have quality metrics. Solution(s) from each hypothesis have also own metrics.
        
            There are main approaches how produce single solution: 
            \begin{itemize}
                \item Solution from best hypothesis. Sorting
                \item Bagging solution
                \item Voting solution                
            \end{itemize}
        
            \paragraph{Designing a Sampling Plan}
             - The most straightforward way of sampling a design space in a uniform fashion is by \cite{EngSurMod}
             means of a rectangular grid of points. This is the full factorial sampling technique referred
             - Latin Squares

        \paragraph{Conclusion}
        For optimization expensive black-box:
        \begin{itemize}
            \item Scalable algorithms that convert multi-objective to single objective problem produce solution that not accurate enough(Scalarizing). Also this approach suitable for a limited type of problem.
            \item Genetic algorithms. This approach is costly to perform and not appropriate for expensive problems.
        \end{itemize}
        Optimization gap in obtaining high quality, multi/single-obj solutions in expensive to evaluate experiments.
        Experiments as a black box, derivative-free. Reference to surrogate optimization.

    \section{Surrogate optimization}
        Surrogate used to expedite search for global optimum. Global accuracy of surrogate
        not a priority. Surrogate model is cheaper to evaluate than the objective.

        Bayesian optimization (BO) methods often rely on the assumption that the objective function 
        is well-behaved, but in practice the objective functions are seldom well- behaved even if 
        noise-free observations can be collected. We propose to address the issue by focusing on 
        the well- behaved structure informative for search while ignoring detrimental structure 
        that is challenging to model data efficiently. [arXiv:1906.11152v2]

        robust surrogate models

        \cite{EngSurMod}

        \paragraph{Use cases}
        Example for each type of optimization. Justification solution.
        Conclusion: Design gap in optimization/parameter tuning. 
        Need to indicate optimization workflow for expensive process/experiments. 
        The argument(s) why we need new architecture. Reference to composition architecture.

    \section{Compositional architecture}
        \paragraph{Interfaces and Contracts}

        \paragraph{Reusable software}
        Problem that each optimization framework/library use inner interfaces. 
        It is necessary to define a standard that implements best practices for extension libraries \cite{buitinck2013api}.

        We introduce new Model-based line for parameter tuning. 

    \section{Scope of work}
        \todo{make some nice tree-diagram}

        Describe and implement workflow for multi-objective parameter tuning of derivative free, black-box system. Parameter estimation is costly.
        The proposed solutions are also suitable for single-criteria optimization. Problem Setting.

        Goal:
        \begin{enumerate}
            \item Globally optimize an objective function(s) that is expensive to evaluate. Single/Multi-objective parameter tuning
            \item Scalability in optimization objective. Simultaneously. Gradient-free evaluation.
            \item Components reuse. Extensibility with other frameworks.
        \end{enumerate}

        Problem:
        \begin{enumerate}
            \item A large number of the target black-box evaluations.
            \item Interfaces not unify.
            \item Code duplication.
        \end{enumerate}

        Solution:
        \begin{enumerate}
            \item Compositional architecture.
            \item Surrogate optimization.
        \end{enumerate}


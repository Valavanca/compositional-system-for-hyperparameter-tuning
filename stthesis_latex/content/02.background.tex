\chapter{Foundation}
    Describe general objectives and there constraints


    This section explains all the technical terms in the last paragraph. To start with, we say that optimizers seek candidate(s) x such that it is 
    unlikely that there exists better candidate(s) y. Each candidate ci is a set of decisions mizers 
    in this article assume the existence of some fitness and their associated objective scores; 
    The optimizers in this article assume the existence of some fitness function f that converts decisions to output objectives; i.e.
    Note that this paper uses the term “model” as a synonym for the fitness function f. 
    Also, the terms single- and multi- objective optimization apply when joj 1 and joj > 1, respectively.

    % -----------------------------------   Parameter tuning
    \section{Parameter tuning}

        Given recent advances in computing hardware, software analysts either validate these models or find 
        optimal solutions by using automatic tools to explore thousands to mil- lions of inputs for their systems. 
        Valerdi notes that, without automated tools, it can take days for human experts to review just a few dozen examples [6]. 
        In that same time, an automatic tool can explore thousands to millions to billions more solutions. 
        People find it an overwhelming task just to certify the correctness of conclusions generated from so many results.

        “..for industrial problems, these algorithms generate (many) solutions, which makes the tasks of under- standing them and selecting one among them difficult and time consuming” [1].



        Parameter Types
        Continuous, integer, ordinal
        Categorical: finite domain, unordered, e.g., {apple, tomato, pepper}

        - Parameter space has structure. Sometimes, some combinations of parameter settings are forbidden
        - Configurations often yield qualitatively different behavior

        Define an objective function: 
        \begin{itemize}
            \item Accuracy
            \item Runtime
            \item Latency
            \item Energy
        \end{itemize}

        Optimization procedures:
        \begin{itemize}
            \item Grid search
            \item Random search
            \item Heuristics
        \end{itemize}
        Bayesian methods differ from random or grid search in that they use past evaluation results to choose the next values to evaluate.
        Limit expensive evaluations of the objective function by choosing the next input values based on those that have done well in the past.

        Optimization cost:
        \begin{itemize}
            \item Evaluation may be very expensive
            \item Sampling budget
            \item Possibly noisy
            \item Feasibility constraints
            \item Multi-objective
        \end{itemize}
        Ideally, we want a method that can explore the search space while also limiting evaluations of poor hyperparameter choices.

        In parameter tuning approaches, a single criterion may not be sufficient to correctly characterize the behavior of the 
        configuration space under consideration and multiple criteria have to be considered.

        One way to simplify the task of understanding the space of possible solutions is to focus on the Pareto frontier; i.e. the subset 
        of solutions that are not worse than any other (across all goals) but better on at least one goal. 
        The problem here is that even the Pareto frontier can be too large to understand. Harman cautions that many frontiers are very crowded; i.e. 
        contain thousands (or more) candidate solutions [7].

        When choosing an optimizer, it is useful to consider what information an optimizer can access about the function f. For example, gradient descent optimizers [16] 
        need to access the contours around every decision. This limits the kinds of functions they can process to those with continuous differential functions (i.e. 
        functions of real-valued variables whose derivative exists at each point in its domain).

    
    \section{Multi-objective optimization}

        Parameter tuning is present in our daily life and comes in a variety of states. The goal is the rich best possible objective by correctly choosing the system parameters. 
        Common of optimization problems requires the simultaneous optimization of multiple, usually contradictory, objectives. These type of problems are termed as multiobjective optimization problems. The solution to such problems is a family of points, that placing on a Pareto front. 


        "Multi-objective optimization deals with such conflicting objectives. It provides a
        mathematical framework to arrive at optimal design state which accommodates the various criteria demanded by
        the application. The process of optimizing systematically and simultaneously a collection of objective functions
        are called multi-objective optimization (MOO) \cite{odugod2013}".
        Thus, a central issue of MOO is the relationship between the objective functions. These are usually modeled as preferences of the decision maker.

        Search or design space(Input space) -> Objective space(Output space).
        Practical optimization problems usually involve simultaneous optimization of multiple conflicting objectives with many constraints(?).

        Knowledge of the Pareto front enables the decision maker to visualize the consequences of his/her choices in terms of performance 
        for a criterion at the expense of one or other criteria, and to make appropriate decisions. Formally, a feasible vector x is said to (Pareto)-dominate another feasible vector x? if
        x is at least as good as x? for all the objectives, and strictly better than x? for at least one objective. The decision vectors in the feasible set 
        that are not dominated by any other feasible vector are called Pareto optimal. The set of non-dominated points in the 
        feasible is the set of Pareto solutions, whose images (by the objective functions) constitute the Pareto front.\cite{Audet2018PerformanceII}


        \subsection{Scalarizing. Weighted sum methods}
            Real problems are generally characterized by the presence of many often conflicting and contradictory objectives. How different objectives should be combined to yield a final solution? There is also the question of how to search for an optimal solution to the design problem. 
            The weighted-sum function approach is a method used to simplify a multiobjective problem, concatenate the objectives into one criterion by using weighted sum factors. The merged objective f is used to evaluate and define the optimal solution.

            One approach which is built on the traditional techniques for generating trade-off surfaces is to aggregate the objectives into a single parameterized objective function.

            Intuitively, Multi-Objective Optimization (MOO) could be
            facilitated by forming an alternative problem with a single,
            composite objective function using weighted sum. Single
            objective optimization techniques are then applied to this
            composite function to obtain a single optimal solution.
            However, the weighted sum methods have difficulties in
            selecting proper weight factors especially when there is no
            articulated a priori preference among objectives. Indeed, a
            posteriori preference articulation is usually preferred, because it
            allows a greater degree of separation between the optimization
            methodology and the decision-making process which also
            enables the algorithmic development process to be conducted
            independently of the application (Giagkiozos et al, 2015).
            Furthermore, instead of a single optimum produced by the
            weighted sum approach, MOO can yield a set of solutions
            exhibiting explicitly the tradeoff between different objectives. \cite{DBLP:journals/corr/abs-1812-07958}


            Given two candidates x; y:? Then x:oi ? y:oi and x:oi ? y:oi is true if objective x:oi ? 
            Each with objectives x:oi; y:oi for 1 ? i ?joj is (worse,better) than y:oi, respectively. to test is one candidate is “better” than another. 
            For multiobjective optimization, determining “better” is somewhat more complicated Traditionally, the space of candidates 
            with multiple objectives was explored by assigning magic weights to the objectives, then using an aggregation function to 
            accumulate the results. Such solutions may be brittle; i.e. they change dramatically if we alter the magic weights of the objectives.

            Why is the Weighting Method Ineffective?[Hirotaka Nakayama]
            Namely, it can not provide a solution among sunken parts of Pareto surface due to “duality gap” for not convex cases. 
            Even for convex cases, for example, in linear cases, even if we want to get a point in the middle of line segment between two vertices, we merely get a vertex of Pareto surface, as
            long as the well known simplex method is used. This implies that depending on the structure of problem, the linearly weighted sum can not necessarily provide a solution as DM desires.


        \subsection{Multy-Objective Evolutionary Algorithms}
            The term evolutionary algorithm (EA) stands for a class of stochastic optimization methods that simulate the process of natural evolution.
            Using strong simplifications, this set is subsequently modified by the two basic principles: selection and variation.
            While selection mimics the competition for reproduction and resources among living beings, the other principle, variation, imitates the natural capability of creating ”new” living beings by means of recombination and mutation. 

            In particular, they possess several characteristics that are desirable for problems involving i) multiple conflicting objectives, and ii) intractably large and highly complex
            search spaces.
            

            Multi-objective Evolutionary Algorithms (MOEAs) are common tools to solve optimization problems, 
            because of their applicability to complex fitness landscapes and solid performance on problems with large design spaces. 
            While other methods also exist, in this thesis we will focus on approaches using Evolutionary Algorithms for the Multy-objective optimizations.

            However, MOEAs still need many evaluations of the "black box" system to solve a typical real-world problem. 
            This is further complicated by the fact that many such problems are very expensive. Consolidated, this makes MOEAs unfeasible for costly and Multy-objective problem.
            A good solution is the integration of the surrogate model which extrapolate and approximate the fitness landscape from samples. MOEA use this surrogate model 
            as a target for optimization. Assumed that solution from surrogate close to a real solution.
            
            We want to understand if the performance of MOEAs approach can be improved by using compositional surrogates. 
            The key idea of compositional surrogates is the splitting objective space to multiple surrogates that extrapolate it independently. 
            Combination of multiple hypotheses should give them the potential to approximate more complicated problems. 

            The various surrogates are analysed on problems of differing complexity, from simple unimodal problems to problems with difficult multimodal. 
            Generating a cloud of candidates may be computationally expensive.

            \begin{itemize}
                \item Quality and Effort tradeoff for multi-objective
                \item Human in the loop: Composition technic as tools for domain expert
            \end{itemize}

            These algorithms eschew the idea of single solutions, preferring instead to use the domination function to map out the terrain of all useful candidates.



            Evolutionary optimizers explore populations of candidate solutions. In each generation some mutator makes changes to the current population. 
            A select operator then picks the best mutants which are then combined in some manner to become generation i 1. 
            This century, there has been much new work on multi-objective evolutionary algorithms with two or three objectives 
            (as well as many-objective optimization, with many more objectives). Recently, there has been much interest in applying MOEAs to many areas 
            of software engineering including requirements engineering, test case planning, software pro- cess planning, etc. 
            This search-based software engineering is a rapidly expanding area of research and a full survey of that work is 
            beyond the scope of this paper (for extensive notes on this topic, see [18], [21]) \cite{MlakarPTF15}.


            \paragraph{Evolutionary Computation}

            Generating the Pareto set can be computationally expensive and is often in- feasible, because the complexity of the underlying application prevents exact methods from being applicable. For this reason, a number of stochastic search strategies such as evolutionary algorithms, tabu search, simulated annealing, and ant colony optimization have been developed: they usually do not guarantee to identify optimal trade-offs but try to find a good approximation, i.e., a set of solutions whose objective vectors are (hopefully) not too far away from the optimal objective vectors \cite{EmmerichD18}.


        \subsection{Metrics for multy-objective solution}

            In single-objective minimization, the quality of a given solution is trivial to quantify:
            the smaller the objective function value, the better. However, evaluating the quality of an approximation of a Pareto set is non trivial.
            The question is important for the comparison of algorithms or prediction next configuration. According

            According to \cite{ZitzlerDT00}, a Pareto front approximation should satisfy the following:
            \begin{itemize}
                \item The distance between the Pareto front and its approximation should be minimized.
                \item A good (according to some metric) distribution of the points of the approximated front is desirable.
                \item The extent of the approximated front should be maximized, i.e., for each objective, a wide range of values should be covered by the non-dominated points.
            \end{itemize}

            The goal of metric is to answer this questions.  

            Metrics for performance indicators partitioned into four groups according to their properties \cite{Audet2018PerformanceII}: 
            \begin{itemize}
                \item cardinality
                \item convergence
                \item distribution
                \item spread
            \end{itemize}
        
            Keep making algorithmic progress toward the Pareto front in the objective function space.
            Straightforward applying of the coefficient of determination (R2) is the wrong indicator of success. 
        
            Evaluations of different sets of Pareto optimal points is multi-objective task.
            Objectives for improving pareto optimal solutions:
            - Keep hypervolume low. Reference point is 0 for all objectives
            - Maximize sparsity of points. Average distance. Crowding Distance
            - Percentage of non-dominant decisions in the total population
        
            Distribution and spread indicators According to \cite{CustodioMVV11}, “the spread metrics try to measure the extents of the spread achieved in a computed Pareto front approximation”. They are not really useful to evaluate the convergence of an algorithm, or at comparing algorithms. They only make sense when the Pareto set is composed of several solutions.
        
            For multi-objective optimization (MOO), an algorithm
            should provide a set of solutions that realize the optimal trade-offs between the considered optimization objectives, 
            i.e., Pareto set. Therefore, the performance comparison of MOO algorithms is based on their Pareto sets.
            In this study, two popular metrics Spacing metric(SM) and Hypervolume (HV) are used to quantify the performance of the algorithms. \cite{DBLP:journals/corr/abs-1812-07958}
           

            \paragraph{Pareto front evaluation}
            \begin{itemize}
                \item Hypervolume (HV)\cite{Zitzler2000ComparisonOM}. 
                This metric represents the volume of the objective space
                that is covered by the individuals of a non-dominated
                solutions set (solutions that belong to a Pareto front). The
                volume is delimited by two points: one point that is called
                the anti-optimal point (A) that is defined as the worst
                solution inside the objective space, and a second optimal
                point (pseudo-optimal) that is calculated by the proposed
                solution method.  
                Determining the hypervolume indicator is a computationally expensive task. 
                Even in case of a reasonably small dimension and low number of points (e.g. 100 points in 10 dimensions), 
                there are currently no known algorithms that can yield the results fast enough for use in most multiple-objective optimizers.

                \item Hyper-area Ratio (HR).
                The Hyper-area Ratio (HR) [24] employs the hypervolume of a solution set A
                divided by the hypervolume value of a Reference Front B. Higher values are
                preferred to lower ones.
                \item Pareto Dominance Indicator (ER). 
                Considers the solutions intersection between two given sets A and B, which can be 
                provided by different algorithms or used to compare a solution set S with a Pareto Front P.
                \item Crowding Distance. *pygmo2. The crowding distance value of a solution provides an estimate of the density of solutions set.
                \item Spacing \cite{Schott1995FaultTD}. Base on this metrics we can say what distribution of Pareto points. Less space metrics means better coverage of population target objectives.
                
            \end{itemize}
        
            Variants in evaluation of sets of solutions for each hypothesis.
            Each hypothesis have quality metrics. Solution(s) from each hypothesis have also own metrics.

            The goal of optimizing an multy-objective problem is to obtain an approximation set A to the PF, including the following two subgoals:
                (1) All solutions in A are as close as possible to the PF,
                (2) All solutions in A are as diverse as possible in the objective space
                (3) Evaluate as few configurations as possible
        
            There are main approaches how produce single solution: 
            \begin{itemize}
                \item Solution from best hypothesis. Sorting
                \item Bagging solution
                \item Voting solution                
            \end{itemize}
        
            \paragraph{Designing a Sampling Plan}
             - The most straightforward way of sampling a design space in a uniform fashion is by \cite{EngSurMod}
             means of a rectangular grid of points. This is the full factorial sampling technique referred
             - Latin Squares

        \paragraph{Conclusion}
        For optimization expensive black-box:
        \begin{itemize}
            \item Scalable algorithms that convert multi-objective to single objective problem produce solution that not accurate enough(Scalarizing). Also this approach suitable for a limited type of problem.
            \item Genetic algorithms. This approach is costly to perform and not appropriate for expensive problems.
        \end{itemize}
        Optimization gap in obtaining high quality, multi/single-obj solutions in expensive to evaluate experiments.
        Experiments as a black box, derivative-free. Reference to surrogate optimization.

    \section{Surrogate optimization}

        To dealing with expensive optimization problem more quickly, we can use surrogate models in the optimization process to approximate the objective functions of the problem. Approximation of solution is faster than the whole optimization process can be accelerated. Nevertheless, the extra time needed to build and update the surrogate models during the optimization process.

        In the literature the term surrogate model (sometimes also meta-model) based optimization is 
        used where, during the optimization processes, some solutions are not evaluated with the original 
        objective function, but are approximated using a model of this function. Different modeling methods 
        are used to build the surrogate models. For single and multiobjective optimization similar methods are used. 
        These methods typically return only one approximated value, which is why in multiobjective problems several 
        models have to be used, so that every model approximates one objective. Some of the most commonly used methods 
        are the Response Surface Method [2], Radial Basis Function [3], Neural Network [4], Kriging [5] and 
        Gaussian Process Modeling [6, 7, 8]. 
        In singleobjective optimization the usage of surrogate models is well established and has proven to be successful. 
        In the literature many different algorithms and various modeling techniques are used to solve real-world and 
        benchmark problems [9, 10]). The results typically show that the surrogate-model-based optimization in 
        comparison with optimization without surrogates provides comparable results in fewer objective function 
        evaluations [11, 12].
        Within surrogate-model-based optimization algorithms a mechanism is needed to find a balance between the 
        exact and approximate evaluations. In evolutionary algorithms this mechanism is called evolution control [13] 
        and can be either fixed or adaptive. In fixed evolution control the number of exact function evaluations that 
        will be performed during the optimization is known in advance. Fixed evolution control can be further divided 
        into generation-based control, where in some generations all solutions are approximated and in the others they 
        are exactly evalu- ated [14], and individual based control, where in every generation some (usually the best) 
        solutions are exactly evaluated and others approximated [15].
        In adaptive evolution control, the number of exactly evaluated solutions is not known in advance,
        but depends on the accuracy of the model for the given problem. Adaptive evolution control can be used in one of two ways: 
        as a part of a memetic search or to pre-select the promising individuals which are then exactly evaluated [16].
        \cite{MlakarPTF15}

        Surrogate used to expedite search for global optimum. Global accuracy of surrogate
        not a priority. Surrogate model is cheaper to evaluate than the objective.

        Bayesian optimization (BO) methods often rely on the assumption that the objective function 
        is well-behaved, but in practice the objective functions are seldom well- behaved even if 
        noise-free observations can be collected. We propose to address the issue by focusing on 
        the well- behaved structure informative for search while ignoring detrimental structure 
        that is challenging to model data efficiently. [arXiv:1906.11152v2]

        robust surrogate models

        In \cite{KrallMD15} proposed approaches that apply kind of surrogate assistant to evaluations and ranging new 
        population. It allows detect the most informative examples in population and evaluate them. 
        identifies and evaluates just those most informative examples
        In the end done less evaluations of real system.Another way to explore solutions is to apply some heuristic to decompose the total space into many smaller problems, and then use a simpler optimizer for each region. For
        Another way to explore solutions is to apply some heuristic to 
        decompose the total space into many smaller problems, and then use a simpler optimizer for each region. 
        
        GP-DEMO \cite{MlakarPTF15} The algorithm is based on the newly defined relations for comparing solutions under uncertainty. 
        These relations minimize the possibility of wrongly performed comparisons of solutions due to inaccurate 
        surrogate model approximations. Using this confidence interval, we define new dominance relations that take into account 
        this uncertainty and propose a new concept for comparing solutions under uncertainty that requires exact evaluations 
        only in cases where more certainty is needed.
        surrogate-model-based MOEA.

        Kind of extend search stage of MOEA with surrogate to simulate evaluation of population. It transform
        problem of searching new better population to improving general hypothesis how and where Pareto set presented.  
        
        We could descreibe compositional-based surrogate optimization as "compound"[ref Asman] gray-box system box black-box optimization
        whit a lot of open research areas where surroagte shound improved, managing portfolio, compare of predictions Pareto fronts,
        As a developer you can focused on specific problem and don't now how implement other components. 
        this is one of the main advantage to the described approach.

        In surrogate-model-based multiobjective optimization, approximated values are often mistakenly used in the 
        solution comparison. As a consequence, exactly evaluated good solutions can be discarded from the population 
        because they appear to be dominated by the inaccurate and over-optimistic approximations. This can slow the 
        optimization process or even prevent the algorithm from finding the best solutions \cite{MlakarPTF15}.

        Surrogates are also used to rank and filter out offspring according to Pareto-related indicators like the 
        hypervolume [20], or a weighted sum of the objectives [21]. The problem with the methods that use 
        hypervolume as a way of finding promising solutions is the calculation time needed to calculate the 
        hypervolume, especially on many objectives. Another possibility is described in [22], where the authors 
        present an algorithm that calculates only non-dominated solutions or solutions that can, because of variance, 
        become non-dominated \cite{MlakarPTF15}.


        \cite{EngSurMod}        

        \paragraph{Use cases}
        Example for each type of optimization. Justification solution.
        Conclusion: Design gap in optimization/parameter tuning. 
        Need to indicate optimization workflow for expensive process/experiments. 
        The argument(s) why we need new architecture. Reference to composition architecture.

        Surrogate based optimization has proven effective in many aspects of engineering and in applications where data is "expensive", or difficult, to evaluate.


    \section{Compositional architecture}
        \paragraph{Compositional surrogates}
        Can the same single-objective models be equally applied to various types of problems in multi-/single-objective optimization?

        When there is no correlation between the outputs, a very simple way to solve this kind of problem is to build n independent models, i.e. one for each output, 
        and then to use those models to independently predict each one of the n outputs. 

        Nevertheless, it is likely that the output values related to the same input are themselves correlated, 
        but an often naive way to build multiple models to capable of predicting simultaneously all n outputs is often given good results. 


        Later research generalized this approach. MOEA/D (multiobjective evolutionary algorithm based on decompo- sition [15]) is a 
        generic framework that decomposes a multi- objective optimization problem into many smaller single problems, 
        then applies a second optimizer to each smaller subproblem, simultaneously

        Moreover, if there are more models, their errors can add up, as well as the time needed to train the models. 
        In memetic algorithms, especially if the surrogate model is not very accurate, a local optimum can be found 
        instead of the global optimum \cite{MlakarPTF15}.


        In the case of pre-selecting the promising individuals, the surrogate model is used to find the promising 
        or drop the low-quality individuals even before they are exactly evaluated, thus reducing the number of 
        exact evaluations. For example, OEGADO [19] creates a surrogate model for each of the objectives. 
        The best solutions in every objective get also approximated on other objectives, which helps with 
        finding trade-off individuals. The best individuals are then exactly evaluated and used to update the models.


        \paragraph{Interfaces and Contracts}


        \paragraph{Reusable software}
        Problem that each optimization framework/library use inner interfaces. 
        It is necessary to define a standard that implements best practices for extension libraries \cite{buitinck2013api}.

        We introduce new Model-based line for parameter tuning. 


    \section{Scope of work}
        \todo{make some nice tree-diagram}

        Describe and implement workflow for multi-objective parameter tuning of derivative free, black-box system. Parameter estimation is costly.
        The proposed solutions are also suitable for single-criteria optimization. Problem Setting.

        Goal:
        \begin{enumerate}
            \item Globally optimize an objective function(s) that is expensive to evaluate. Single/Multi-objective parameter tuning
            \item Scalability in optimization objective. Simultaneously. Gradient-free evaluation.
            \item Components reuse. Extensibility with other frameworks.
        \end{enumerate}

        Problem:
        \begin{enumerate}
            \item A large number of the target black-box evaluations.
            \item Interfaces not unify.
            \item Code duplication.
        \end{enumerate}

        Solution:
        \begin{enumerate}
            \item Compositional architecture.
            \item Surrogate optimization.
        \end{enumerate}

\chapter{Introduction}\label{sec:intro}

\begin{blockquote}
\paragraph{Intent:} A short version of thesis and a description of done work. Challenges and Problems.

    \begin{description}
        \item[1. Motivation] Surrogate model for multi-objective expensive black-box problem $\rightarrow$ Research gap: Portfolio/Compositional system/Sampling plan. Definition and motivation of the goal. Goal: MO solution $\rightarrow$ Problem: Expensive black-box $\rightarrow$ Solution: Answer research questions
        \item[2. Objectives of work] ?
        \item[3. Research Questions] Question from research gap. The answer to this questions is the purpose of the thesis
        \item[4. Results overview] A short overview of done work
    \end{description}
\end{blockquote}

% --------------------------------------------------------------------------------------------
% ------------------------------------------------     Motivation      
% --------------------------------------------------------------------------------------------
\section{Motivation}

    The central purpose of this thesis is to investigate portfolio of surrogate models that can be used to improve applicability model-based optimization methods to a verity of problems such as parameter tuning. Surrogate model or models based optimization is a common approach for a deal with expensive black-box function, but as far as the author is aware, there is no published research where the influence of heterogeneous portfolio of surrogate models was studied. The main target problem is an expensive multi-objective problem but the developed approach is also suitable for expensive single-objective optimization.
    As black-box, we can not say what type of surface does the problem have. That is why it should be customized in the optimization process. The goal is to determine if the variability in extrapolation worth it. Introduce new surrogate-design-criteria for multi-objective hyperparameter optimization software.

    It also provides backward compatibility for a single-objective problem. This optimization approach can significantly reduce expensive evaluations counts but torment from problems such as sampling size, type of surface and optimization techniques. We developed and adopted new technic in MBO such as portfolio surrogates, compositional model and surrogate validation. 

    Multi-objective optimisation is an established parameter tuning technique. It is especially suited to solve complex, multidisciplinary design problems with an accent on system design.

    When we talk about several objectives, the intention is to find good compromises rather than a single solution as in global optimization.
    Since the solution for multi-objective optimization problems gives the appearance to a set of Pareto-optimal points, evolutionary optimization algorithms are ideal for handling multi-objective optimization problems.

    General optimization methods could be classified into derivative and non-derivative methods. In this thesis focuses on non-derivative methods, as they are more suitable for parameter tuning. Therefore, they are also known as black-box methods and do not require any derivatives of the objective function to calculate the optimum.  Other benefits of these methods are that they are more likely to find a global optimum. 


% --------------------------------------------------------------------------------------------
% ------------------------------------------------     Objectives      
\section{Objectives}
    Black-box multi-objective problems given a finite number of function evaluations

% --------------------------------------------------------------------------------------------
% ------------------------------------------------     Research Questions      
\section{Research Questions}


\begin{description}
    \item[RQ1:\label{RQ1}] Heterogeneous surrogate models for multiobjective optimization
    \item[RQ2:\label{RQ2}] Domain independent sampling strategies
    \item[RQ3:\label{RQ3}] Scalable surrogate-based optimization 
\end{description}

Addition:

\begin{description}
    \item[RQ1(Surrogate portfolio):] How a surrogate portfolio influence on the optimization process?
    \item[RQ2(Composition model):] How a compositional heterogeneous surrogate model influence on the optimization process?
    \item[RQ3(Sampling plan):] Dynamic sampling plan on arbitrary problem
\end{description}

% --------------------------------------------------------------------------------------------
% ------------------------------------------------     Overview
\section{Results overview}
In numerous test problems, compositional-surrogate finds comparable solutions to standard MOEA (NSGA-II, MOEAD, MACO, NSPSO) doing considerably fewer evaluations (300 vs 5000). 
Surrogate-based optimization is recommended when a model is expensive to evaluate.

% \section{Solution}

% \section{Organization of the Thesis}

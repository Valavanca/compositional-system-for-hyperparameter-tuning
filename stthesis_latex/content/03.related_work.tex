\chapter{Related work}\label{sec:related}

    \begin{blockquote}
        \paragraph{Intent:} This section overview other studies in the area of surrogate-based multi-objective optimization or related approach from other types of optimization. The gaps of the others should be clearly. 
        \todor{Add many objective related work}
        Structure:
        \begin{description}
            \item[Comparison criteria] Surrogate type, Surrogate portfolio, Solver type, Sampling size, Many-objective optimization
            \item[Short description] 
            \item[Comparable table] 
        \end{description}
    \end{blockquote}


    % --------------------------------------------------------------------------------------------
    % ------------------------------------------------          Criteria
    % --------------------------------------------------------------------------------------------
    \subsection{Comparison criteria}
        Many existing approaches can be categorized as multi-objective optimization. That is why introduce comparison criteria for a clear and concise definition of the approach presented in this thesis:
        \begin{description}
            \item[Surrogate type] Extrapolation technique for a surrogate model. Surrogate combination
            \item[Sampling plan] Collecting sampling points for building a surrogate model
            \item[Optimization] Algorithm to find a multi-objective solution(s)
            \item[Scalability] Many-objectivity. Problems with high dimensionality in objective and parameter spaces                
        \end{description}
        Also, there are other important features such as categorical variables, prior knowledge, feasibility constraints which in most cases are not implemented.
        
        Almost all related works of parameter tuning could be categorized as sequential model-based optimization (SMBO)\cite{JonesSW98}.
        The general SMBO workflow:
        \begin{enumerate}
            \item The initial samples plan of evaluation points
            \item Fit a regression model to provide a hypothesis on the relation between parameters and objectives
            \item Use the build surrogate model as a parameter for optimization strategy. Generate solutions, new promising point
            \item Evaluate the new point with the black-box function and add it to the samples
        \end{enumerate}
        We will present an overview of available works for model-based multi-objective optimization. Beginning with an overview of the project for model-based optimization (mlrMBO), and continue with related algorithms including various optimization improvements.


        % Note that this approach means that

    % --------------------------------------------------------------------------------------------
    % ------------------------------------------------          Approaches
    % --------------------------------------------------------------------------------------------

    Bayesian optimization (BO) methods often rely on the assumption that the objective function is well-behaved, but in practice, the objective functions are seldom well-behaved even if noise-free observations can be collected. In \cite{bodin2019modulating} propose robust surrogate models to address the issue by focusing on the well- behaved structure informative for search while ignoring detrimental structure that is challenging to model data efficiently.


    % ------------------------------------------------          Frameworks / Platforms
    \subsection{Platforms and frameworks}
        There are a lot of different projects that can handle multi-objective solutions. Frameworks present multiple multi-objective algorithms, performance metrics, build in test problems, and extra tools such as plotting and benchmarks. Some frameworks focused on efficient calculations and parallelization\cite{francesco_biscani_2019}, other on implementation modern multi-objective algorithm\cite{benitezhidalgo2019jmetalpy, TianPlatEMO} or achieving plenty model-based optimization algorithms\cite{BischlmlrMBO}. 

        % ----------------         mlrMBO
        \paragraph{mlrMBO}\cite{BischlmlrMBO} is modular framework for model-based optimization of expensive black-box functions. MlrBO extends the MBMO procedure for mixed and hierarchical parameter spaces. For the surrogate, that project allows any regression learner from $mlr$ library. That is why a bagging method can be applied to regression models to retrieve model error in $mlr$. This framework enables prediction several points for evaluation. However, it doesn't provide a combination of different surrogates into one model. Algorithms implementation is static
       
        % ?General characteristics are that they have multiple algorithms on each type of optimization and additional features that they provide. Usually, a surrogate model is predefine and injected into some algorithm for decision making.

        % ----------------         BRISE 2.0
        \paragraph{BRISE 2.0} \cite{PukhkaievG18} is software product line for parameter tuning. Some features of their approach are distributed architecture, reusable components and complex stop condition. Extrapolation model for parameter prediction is exchangeable and combines surrogate and optimization strategy in each implementation. There are implemented several general models such as polynomial regression surrogate with local search and Bayesian optimization with bandit-based methods strategy(BOHB \cite{FalknerBOHB}). In $BRISE 2.0$ extend the idea of model validation to exchange static sapling plan. During optimization, the final configuration predicts from the sampling plan if the model does not have enough R2 score or energy prediction less than 0. The main advantage of dynamic sampling plan is that it does not require initial knowledge about the problem. However, the models implemented statically and should be adopted to concrete domain-problem.



    % ------------------------------------------------          Algorithms
    \subsection{Model-based multi-objective algorithms}
        Fixed optimization components can make general optimization ineffective in the presence of changeable problems. Surrogate construction methodology or even MO algorithm could change to a rich solution closest to real Pareto-optimal frontier.

        % ----------------         SMAC
        \paragraph{Sequential Model-based Algorithm Configuration (SMAC)} SMAC\cite{smac-2017} adopted a random forest model and Expected Improvement (EI) to model a conditional probability. It applies a local search with several starting points and picks configurations with maximal EI. EI improves the exploration possibilities of SMAC on points with larger uncertainty and an optimal value of the objective mean. However, SMAC is limited to single-criteria optimization and uses a predefined sampling plan.


        % ----------------         ParEGO
        \paragraph{ParEGO} scalarization based multi-objective algorithm \cite{Knowles06}. The classical single-objective algorithm $EGO$ of Jones et al.  was extended to a multi-point proposal by repeat algorithm several times with randomly change scalarization weights in each iteration.  The idea of algorithm based on Kriging/Gaussian process regression model and multiple single objective optimization processes on scalarized objectives guaranteeing that multiple points on the Pareto-optimal front could be in prediction. This algorithm and its modification implemented in mlrMBO\cite{BischlmlrMBO}.


        % ----------------         Distributed Surrogates
        \paragraph{An Evolutionary Algorithm with Spatially Distributed Surrogates for Multiobjective Optimization} Amitay et al.,\cite{DistrSurr}presented in their paper an evolutionary algorithm with spatially distributed surrogates (EASDS). Surrogate models used Radial Basis Function Networks, periodically validation and updating for each partition from samplings points. Spatially Distributed Surrogate models are created for all the objectives and then evaluated by NSGA-II. The authors describe that their approach achieves better results than single global surrogate models showing an advantage from using multiple surrogates. However, the authors evaluated there algorithm only on bi-objective problems

        % ----------------         Hybrid surrogate
        \paragraph{A hybrid surrogate-based approach for evolutionary multi-objective optimization} Rosales-Pérez et al.,\cite{HybridSurrRCG} proposed an approach based on an ensemble of Support Vector Machines. The authors describe a model selection process or hyperparameters selection of SVM based on validation technique and fother injection to the surrogate ensemble. Incremental development of the ensemble includes new information obtained during the optimization and stores previous models. The training of a new model carries the grid search of SVM Kernel types to find one with the least expected generalization error. This paper presents a model selection process for determining the hyperparameters for each SVM in the ensemble.

        % ----------------         Surrogate Search
        \paragraph{Efficient Multi-Objective Optimization through Population-based Parallel Surrogate Search} Akhtar et al.,\cite{akhtar2019efficient} introduce a multi-objective optimization algorithm for expensive functions that connect iteratively updated several surrogates of the objective functions. The feature of this algorithm is high optimization for parallel computation. An algorithm combines Radial Basis Function (RBF) approximation, Tabu, and local search around multiple points. Authors present an algorithm that can theoretically be applicable for hight dimensional space and many-objective problems.

        % ----------------         Hierarchical surrogates
        \paragraph{Evolutionary optimization with hierarchical surrogates} Xiaofen Lu et al. \cite{LuST19} apply different surrogate modeling techniques based on motivation on optimization expensive black-box function without any prior knowledge on a problem. They used a pre-specified set of models to construct hierarchical surrogate during optimization. Also, for verification of surrogate used general accuracy of the high-level model. The whole process of the proposed method split to accumulate training samples and model-based optimization, which means that the sample plan is static and requires prior information about the problem.
        %! A mathematic analysis was given in this paper to show that the hierarchical surrogate structure can be beneficial when the accuracy of the high-level model is larger than 0.5.
        % !However, one modeling technique might perform differently on different problem landscapes.

        % ----------------         GALE
        \paragraph{GALE: Geometric Active Learning for Search-Based Software Engineering} Krall et al.,\cite{KrallMD15} developed an algorithm that uses PCA and active learning techniques to step-by-step approximation and evaluating the most informative solutions. The main features of GALE are active learning in geometrically analysis perspective regions in search space to select most prospective candidates.

        % ----------------         Hypermapper
        \paragraph{Hypermapper} Luigi Nardi et al. \cite{nardi2019practical} presented a multi-objective black-box optimization tool. Some features of their approach are prior knowledge, categorical variables, feasibility, and excellent adaptation to embedded devices. They train separate models, one for each objective and constraints. Then merge it with random scalarization (Tchebyshev/ Linear scalarization). Next, the Bayesian model leads local search algorithm to explore Pareto optimal points. 

        % ?? ----- GP-DEMO
        GP-DEMO \cite{MlakarPTF15} authors present algorithm is based on comparing solutions under uncertainty. Using this confidence interval, we define new dominance relations that take into account this uncertainty and propose a new concept for comparing solutions under uncertainty that requires exact evaluations only in cases where more certainty is needed.

        \begin{table}[]
            \caption{Related work: frameworks and algorithms\label{related_w}}
            \centering
            \resizebox{\textwidth}{!}{%
            \begin{tabular}{@{}llllll@{}}
                                                & Surrogate                             & Optimization Algorithm      & Sampling plan & Scaling of test problems & Year \\  \hline
            Distributed Surrogates (EASDS) & Radial Basis Function Networks, clustering & NSGA-II                                            & Static: Rand & 2 obj & 2007 \\
            Hybrid surrogate-based approach     & Ensemble of Support Vector Machines   & NSGA-II                     & Static: lhs   & 2 obj                    & 2013 \\
            mlrMBO                              & Regression model(s)                   & Single-/Multi-opt algorithm & Static        & 2 obj                    & 2017 \\
            BRISE 2.0                           & Polinomial regression, BOHB           & local search, BOHB          & Dinamic       & 1 obj                    & 2018 \\
            SMAC                                & Random forests + Expected Improvement & Local search                & Static        & 1 obj                    & 2017 \\
            Parallel Surrogate Search (MOPLS-N) & Radial Basis Function                 & Tabu and local search       & Static        & 2 obj                    & 2019 \\
            GALE                                & PCA-approximation + linear models     & MOEA/D                      & Static        & 2-8 obj                  & 2015 \\
            Hypermapper                    & Randomized decision forests                & Scalarizations + Gaussian Processes + Local search & Static       & 2 obj & 2019 \\
            ParEGO                              & Kriging/Gaussian process regression   & scalarizations + EGO        & Static: lhs   & 2-3 obj                  & 2006 \\
            Hierarchical surrogates             & Dinamic hierarchical surrogate        & EA                          & static        & 1 obj                    & 2019 \\
            BOHB                                & tree-structured Parzen estimator      & Bandit learning             & static        & 1 obj                    & 2018
            \end{tabular}%
            }
        \end{table}

    % --------------------------------------------------------------------------------------------
    % ------------------------------------------------          Conclusions
    % --------------------------------------------------------------------------------------------
    \section{Conclusions}

        To our knowledge, there exists no research that studies a complex combination of the surrogate model and how it affects on solution quality.  Fixed optimization components can make general optimization ineffective in changeable problems. Appropriate surrogate construction methodology or combination of multiple solving algorithms could improve the final solution. The significant works analyze improving surrogate architecture as a promising direction. Regardless they operate with a limited amount of models and implementation is inflexible. Moreover, almost none of the approaches examined the size of the surrogate sample set. 
        Some of the important conclusions from related work \cite{SoftSurvey}:
        \begin{enumerate}
            \item The optimization result could be improved with a precise selection of surrogate architecture
            \item Most of the algorithms were based on dominance-based evolutionary algorithms
            \item Most of the algorithms solved the problems with no more than three objectives and limited decision variables
            \item Sampling plan is static
        \end{enumerate}


        After an analysis we can define research gaps such as:
        \begin{description}
            \item[Surogate combination] Previous researches have shown that the selection of the surrogate model can profoundly affect the quality of the solution from the optimization algorithm. \cite{LuST19} mentioned that surrogate is domain-specific, and the same technique might perform differently on different problems. Therefore, a technology that allows to combine and scale multiply models flexibly is necessary.
            \item[Sampling plan] Independent sampling plan from a problem can lead the search in the wrong directions. This problem may occur because the surrogate model may be under fitted or overfitted, and the result is a waste of samples and resources.
        \end{description}

        As shown before, surrogate or model-based optimization suited for an expensive black-box problem. Nevertheless, this approach also has open questions and limitations:
        \begin{itemize}
            \item \emph{Surrogate is domain-specific.} For improving and reach the best prediction, we should know in cross-grain underlying objective surface to apply specific surrogate. Universal surrogates can gain optimal results but not the most reliable possible \cite{abs181207958, LuST19}.
            \item Multi-objective hypothesis. A \emph{limited amount} of surrogate models can handle multi-dimensional parameter space. Also, scaling is an open research question for existing multi-output surrogate models.
            \item Quality of prediction depends on \emph{how much samples} do we have for a specific type of surrogate. There is a trade-off between reducing sample size and maximize the quality of prediction. Overfitting, as well as Underfitting, can guide optimization in the wrong direction.
            \item Multi-objective optimizations with \emph{categorical features} are standard in engineering practices. Parameter tuning with that type of space is not trivial.
            \item Often Optimization algorithm and surrogate model are very \emph{tightly coupled}. Profoundly connected implementation suitable for a specific type of problem and require prior knowledge about a problem. Reimplementing these algorithms for each usage scenario becomes timeconsuming and error-prone. 
        \end{itemize}



        % !There is a clear need for a method to provide and distribute ready-to-use implementations of optimization methods and ready-to-use benchmark and real-world problems. These modules should be freely combinable. Since the above mentioned issues are not constrained to evolutionary optimization a candidate surrogates solution should be applicable to a broader range of search algorithms



        % However, it lacks fundamental features that makes it ineffective in the presence of applications with non-feasible designs and prior knowledge.


        % Our work is similar in nature to the approaches adopted in the Bayesian optimization literature [27]. Example of widely used mono-objective Bayesian DFO software are SMAC [15], SpearMint [30], [31] and the work on tree-structured Parzen estimator (TPE) [6]. In particular, the use of a full Bayesian ap- proach will help to leverage the prior knowledge by computing a posterior distribution. Exploration of additional methods to warm-up the search from the design of experiments literature

        % \paragraph{Hyperopt}




        % Based on a survey\cite{SoftSurvey} there still exist such as:
        % \begin{itemize}
        %     \item 
        % \end{itemize}


        % Some features of their approach are 


        % We extend the idea of stochastic RBF to be suitable for the algorithm configuration task. It is a model-based algorithm that cycles from emphasis on the objective to emphasis on the distance using a weighting strategy. 
        
        % The main advantage of LHS is that it does not require an increased initial population size for more dimensions.


        % exploration, exploitation and diversification during each algorithm iteration.


        % To our knowledge, there exists no research that investigates how deep networks affect the performance of surrogate-assisted multi-objective evolutionary algorithms. If neural networks were used, the approaches usually adopted only one hidden layer. Additionally, we can also see that there exists a multitude of ways a surrogate model can be integrated into an evolutionary algorithm

        % Previous studies have shown that the choice of modeling technique can highly affect the performance of the surrogate model-assisted evolutionary search.

        % However, one modeling technique might perform differently on different problem landscapes. Without


        % OpenTuner is different from our work in a number of ways. First, our work supports multi-objective optimization. Second, our white-box model-based approach enables the user to understand the results while learning from them. Third, our approach is able to consider unknown feasibility constraints. Lastly, our framework has the ability to inject prior knowledge into the search. The first point in particular does not allow a direct performance comparison of the two tools.

        % over the entire space would result in a waste of samples.


        % This is especially the case for a SMBO

        % The black-box function can be more complex, for example a machine learning pipeline which includes preprocessing, feature selection and model selection.


        % Consequently, the advantage of the multi-objective candidate generation to produce a set instead of single points is not only used for improving the exploration of the decision space, but also for obtaining a well-spread batch of solutions.


        % For example, OEGADO \cite{ChafekarSRX05} creates a surrogate model for each of the objectives. The best solutions in every objective get also approximated on other objectives, which helps with finding trade-off individuals. The best individuals are then exactly evaluated and used to update the models.
\chapter{Implementation. Development}
Managing Complex Execution Strategies

\section{Portfolio with hypothesis}
A set of models is defined that can form a partial or complete hypothesis to describe the problem.
Also during the increase of the experiments may change the model that best describes the existing problem
As a result, there is variability for each problem and configuration step at the same time. 
A set of hypotheses can solve this problem but it takes longer time for cross validation.

\section{Validate hypothesis}
    \epigraph{``All models are wrong but some are useful``}{\textit{– George Box}}

    \subsection{Sampling strategy}
    Oversampling and undersampling in data analysis. Alleviate imbalance in the dataset. 
    Imbalance in dataset is not always a problem, more so for optimization tasks. 

    The main gain for models not to provide best accuracy on all search space but provide possible optimum regions.
    Accuracy in prediction optimal regions or points from there will direct the search in the right direction.

    Predictor variables can legitimately over- or under-sample. 
    In this case, provided a carefully check that the model assumptions seem valid.

\section{Acquisition in points for evaluation}

    \subsection{Metrics}
    Keep making algorithmic progress toward the Pareto front in the objective function space.
    Coefficient of determination (R2) is the wrong indicator of success.

   

    \paragraph{How evaluated a Pareto front?}
    \begin{itemize}
        \item Hypervolume (HV). 
        This metric represents the volume of the objective space
        that is covered by the individuals of a non-dominated
        solutions set (solutions that belong to a Pareto front). The
        volume is delimited by two points: one point that is called
        the anti-optimal point (A) that is defined as the worst
        solution inside the objective space, and a second optimal
        point (pseudo-optimal) that is calculated by the proposed
        solution method. 
        \item Hyper-area Ratio (HR).
        The Hyper-area Ratio (HR) [24] employs the hypervolume of a solution set A
        divided by the hypervolume value of a Reference Front B. Higher values are
        preferred to lower ones.
        \item Pareto Dominance Indicator (ER). 
        Considers the solutions intersection between two given sets A and B, which can be 
        provided by different algorithms or used to compare a solution set S with a Pareto Front P.
        \item Crowding Distance. *pygmo2
        
    \end{itemize}

    Variants in evaluation of sets of solutions for each hypothesis.
    Each hypothesis have quality metrics. Solution(s) from each hypothesis have also own metrics.

    There are main approaches how produce single solution: 
    \begin{itemize}
        \item Solution from best hypothesis. Sorting
        \item Bagging solution
        \item Voting solution                
    \end{itemize}

    \paragraph{Designing a Sampling Plan}
     - The most straightforward way of sampling a design space in a uniform fashion is by \cite{EngSurMod}
     means of a rectangular grid of points. This is the full factorial sampling technique referred
     - Latin Squares
\section*{Abstract}
    The multi-objective decision making is critical for an everyday task and also for engineering problems. Find perfect trade-off to enhance all criteria requires much experience or the availability of a significant amount of resources; that it is not feasible to achieve for an expensive problem such as engineering. The state-of-the-art approach is model-based or surrogate-based optimization that used approximation models of the real problem that is cheap to evaluation. These models are a simplified hypothesis of cause-effect relationships and replace high estimates with cheap approximations. In this thesis, we used surrogate models as wrappers on the real problem and applied \gls{moea} to detect  Pareto optimal decisions. 
    
    The general idea is combining and stacking several models that describe multiple objective sub-space independently and optimize it as a single surrogate hypothesis - surrogate compositional model. Combination of multiple models gave potential to approximate more complicated problems and stacking of valid surrogate hypothesis speed-up convergence. Accordingly, a better result is obtained at lower costs.
    We combine several possible surrogate variants and use those that pass validation. After recombination of valid single objective surrogates to a multi-objective surrogate hypothesis, several instances of \gls{moea}s provide several Pareto-front approximations. The modular structure of implementation allows us to avoid static sampling plan, use self-adaptable models with customizable portfolio. In numerous case studies, our methodology finds comparable solutions to standard NSGA2 using considerably fewer evaluations. The developed approach is recommended for parameter tuning of expensive black-box functions.
\section*{Abstract}
    Multi-objective decision-making is critical for everyday tasks and engineering problems. Finding the perfect trade-off to maximize all the solution's criteria requires a considerable amount of experience or the availability of a significant number of resources. This makes these decisions difficult to achieve for expensive problems such as engineering. Most of the time, to solve such expensive problems, we are limited by time, resources, and available expertise. Therefore, it is desirable to simplify or approximate the problem when possible before solving it. The state-of-the-art approach for simplification is model-based or surrogate-based optimization. These approaches use approximation models of the real problem, which are cheaper to evaluate. These models, in essence, are simplified hypotheses of cause-effect relationships, and they replace high estimates with cheap approximations. In this thesis, we investigate surrogate models as wrappers for the real problem and apply \gls{moea} to find Pareto optimal decisions. 
        
    The core idea of surrogate models is the combination and stacking of several models that each describe an independent objective. When combined, these independent models describe the multi-objective space and optimize this space as a single surrogate hypothesis - the surrogate compositional model. The combination of multiple models gives the potential to approximate more complicated problems and stacking of valid surrogate hypotheses speeds-up convergence. Consequently, a better result is obtained at lower costs.
    We combine several possible surrogate variants and use those that pass validation. After recombination of valid single objective surrogates to a multi-objective surrogate hypothesis, several instances of \gls{moea}s provide several Pareto front approximations. The modular structure of implementation allows us to avoid a static sampling plan and use self-adaptable models in a customizable portfolio. In numerous case studies, our methodology finds comparable solutions to standard NSGA2 using considerably fewer evaluations. We recommend the present approach for parameter tuning of expensive black-box functions.
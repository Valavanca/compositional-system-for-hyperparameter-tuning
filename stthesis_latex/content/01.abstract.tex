\chapter{Introduction. Challenges and Problems.}

[DLR, Optimisation-based multi-objective design and assessment] 
Multi-objective optimisation is a proven, well-known parameter tuning technique in engineering design. It is especially suited 
to solve complex, multidisciplinary design problems with emphasis on control system design.
MOPS is currently applied to various design and evaluation problems at DLR and in industry. The main fields 
of application are industrial robotics, flight control, power-optimized aircraft systems, and vehicle dynamics. Development and maturation of MOPS is an ongoing process.
In MOPS a multi-objective/multi-model/multi-case design problem is usually mapped to a weighted min-max optimisation problem, which is then solved by using one of several 
available powerful optimizers, implementing local and global search strategies. Besides very efficient gradient-based solvers (well-suited primarily for smooth problems, 
especially identification problems), more robust gradient-free direct-search based solvers are available to address problems with non-smooth or noisy criteria. To overcome the 
problem of local minima to some extent, global solvers based on stochastic, evolutionary or branching strategies can be alternatively used.

The weighted-sum function approach is a method used to simplify a multiobjective problem, lumping the objectives into 
one function by using weighted sum factors. The combined function f is used to evaluate and define the optimal solution.

Real engineering design problems are generally characterized by the presence of many
often conflicting and incommensurable objectives. This raises the issue about how
different objectives should be combined to yield a final solution. There is also the
question on how to search for an optimal solution to the design problem. 

When we talk about several objectives, the notion of optimum changes because in multiobjective problems, the aim is to find good compromises rather than a single solution as in global optimization.
Since multi-objective optimization problems give rise to a set of Pareto-optimal solutions, evolutionary 
optimization algorithms are ideal for handling multi-objective optimization problems.

[Johan Andersson, A Survey of Multiobjective Optimization in Engineering Design] 
Optimization methods could be divided into derivative and non-derivative methods. This survey focuses on non-derivative methods, as they are more suitable for
general engineering design problems. One reason is that non-derivative methods do not
require any derivatives of the objective function in order to calculate the optimum.
Therefore, they are also known as black box methods. Another advantages of these
methods are that they are more likely to find a global optima, and not be stuck on local
optima as gradient methods.

\section{Motivation}
Motivation Examples

\section{Objectives}

\section{Research Questions}
\begin{enumerate}
    \item RQ: How reduce experiments count and reach near-optimal multi objective solution?
    \item RQ: Reusable compositional system for optimization. Define steps in workflow. How extend models to new use case and not to rewrite everything from scratch.
    \item RQ: No free lunch theorem: hypothesis portfolio. Select from a plethora of models that can be suitable for fitted data set. Usefully for single and multi objective in parameter tuning.
    \item RQ: Solve hypothesis and range multi-objective solution
\end{enumerate}

% \section{Solution}

% \section{Organization of the Thesis}

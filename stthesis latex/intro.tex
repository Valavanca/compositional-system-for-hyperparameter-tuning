\documentclass[ms,english]{stthesis}
\usepackage{lipsum}

\title{Compositional multi-objective parameter tuning}
\author{Oleksandr Husak}
\date{\today}
\birthday{16.04.1994}
\birthplace{Ukraine}
\supervisor{Dr.-Ing. Birgit Demuth}


\begin{document}
    \maketitle % This sets the title page
  
    \tableofcontents
  
    \chapter{Introduction. Black-box Optimization}
        [DLR, Optimisation-based multi-objective design and assessment] 
        Multi-objective optimisation is a proven, well-known parameter tuning technique in engineering design. It is especially suited 
        to solve complex, multidisciplinary design problems with emphasis on control system design.
        MOPS is currently applied to various design and evaluation problems at DLR and in industry. The main fields 
        of application are industrial robotics, flight control, power-optimised aircraft systems, and vehicle dynamics. Development and maturation of MOPS is an ongoing process.
        In MOPS a multi-objective/multi-model/multi-case design problem is usually mapped to a weighted min-max optimisation problem, which is then solved by using one of several 
        available powerful optimisers, implementing local and global search strategies. Besides very efficient gradient-based solvers (well-suited primarily for smooth problems, 
        especially identification problems), more robust gradient-free direct-search based solvers are available to address problems with non-smooth or noisy criteria. To overcome the 
        problem of local minima to some extent, global solvers based on stochastic, evolutionary or branching strategies can be alternatively used.

        The weighted-sum function approach is a method used to simplify a multiobjective problem, lumping the objectives into 
        one function by using weighted sum factors, as shown in Eq. [14.3]. The combined function f is used to evaluate and define the optimal solution.

        Real engineering design problems are generally characterized by the presence of many
        often conflicting and incommensurable objectives. This raises the issue about how
        different objectives should be combined to yield a final solution. There is also the
        question on how to search for an optimal solution to the design problem. 

        When we talk about several objectives, the notion of optimum changes because in multiobjective problems, the aim is to find good compromises rather than a single solution as in global optimization.

        [Johan Andersson, A Survey of Multiobjective Optimization in Engineering Design] 
        Optimization methods could be divided into derivative and non-derivative methods,Figure 2. This survey focuses on non-derivative methods, as they are more suitable for
        general engineering design problems. One reason is that non-derivative methods do not
        require any derivatives of the objective function in order to calculate the optimum.
        Therefore, they are also known as black box methods. Another advantages of these
        methods are that they are more likely to find a global optima, and not be stuck on local
        optima as gradient methods might do

        \section{Motivation}
        Why is the Weighting Method Ineffective?[Hirotaka Nakayama]
        Namely, it can not provide a solution among sunken parts of Pareto surface due to “duality gap” for nonconvex cases. 
        Even for convex cases, for example, in linear cases, even if we want to get a point in the middle of line segment between two vertices, we merely get a vertex of Pareto surface, as
        long as the well known simplex method is used. This implies that depending on the structure of problem, the linearly weighted sum can not necessarily provide a solution as DM desires.

      
        \section{Objectives}
        \lipsum
  	
        \section{Overview}
            \begin{enumerate}
                \item RQ: How reduce experiments count and reach near-optimal multi objective solution?
                \item RQ: Reusable compositional system for optimization. Define steps in workflow. How extend models to new use case and not to rewrite everything from scratch.
                \item RQ: No free lunch theorem: hypothesis portfolio. Select from a plethora of models that can be suitable for fitted data set. Usefully for single and multi objective in parameter tuning.
                \item RQ: Solve hypothesis and range solution
            \end{enumerate}

            Since multi-objective optimization problems give rise to a set of Pareto-optimal solutions, evolutionary optimization algorithms are ideal for handling multi-objective optimization problems [1].
    
    \chapter{Foundation}
        \section{Parameter tuning}
        \section{Multi-objective optimization}
        \section{Surrogate optimization}
            Surrogate used to expedite search for global optimum. Global accuracy of surrogate
            not a priority
        \section{Use cases}

    \chapter{Compositional architecture}
        \section{Interfaces and Contracts}           
        \section{Reusable software}

% --- IMPLEMENTATION 
    \chapter{Implementation}
        Managing Complex Execution Strategies
        \section{Portfolio with hypothesis}
        \section{Validate hypothesis and solve them}
        \section{Compute best next point (range of points)}

            \subsection{Metrics}
                Keep making algorithmic progress toward the Pareto front in the objective function space;
                \paragraph{Overview}
                \paragraph{How evaluated a Pareto front?}
                    - Hypervolume
                    - Hyper-area Ratio (HR)
                    The Hyper-area Ratio (HR) [24] employs the hypervolume of a solution set A
                    divided by the hypervolume value of a Reference Front B. Higher values are
                    preferred to lower ones.

                    - Pareto Dominance Indicator (ER) [11] considers the solutions intersection between two given sets A and B, which can be 
                    provided by different algorithms or used to compare a solution set S with a Pareto Front P.

                    - Crowding Distance

            \begin{itemize}
                \item Solution from best hypothesis
                \item Bagging solution
                \item Voting solution                
            \end{itemize}


% --- EVALUATION 
    \chapter{Evaluation. Experimental Results}

            Will be used two types of problems: Synthetic and Real physical

            Idea: Generate problem from data set and try to optimize it with parameter tunning from the beginning. Need models with accuracy ~96 amd multiply objectives. 

            \section{Test suite: ZDT}
            This widespread test suite was conceived for two-objective problems and takes its name from its authors Zitzler, Deb and Thiele.
            Ref[“Comparison of multiobjective evolutionary algorithms: Empirical results.”, 2000]

            \section{Test suite: DTLZ}
            This widespread test suite was conceived for multiobjective problems with scalable fitness dimensions and takes its name from its authors Deb, Thiele, Laumanns and Zitzler.
            Ref["Scalable Test Problems for Evolutionary Multiobjective Optimization", 2005]

            \section{Test suite: WFG}
            This test suite was conceived to exceed the functionalities of previously implemented test suites.
            In particular, non-separable problems, deceptive problems, truly degenerative problems and mixed shape Pareto front problems are thoroughly covered, 
            as well as scalable problems in both the number of objectives and variables. Also, problems with dependencies between position and distance related parameters are covered.
                \begin{enumerate}

                    \item A few unimodal test problems should be present in the test suite. 
                    Various Pareto optimal geometries and bias conditions should define these problems, 
                    in order to test how the convergence velocity is influenced by these aspects.

                    \item The following three Pareto optimal geometries should be present in the test suite: 
                    degenerate Pareto optimal fronts, disconnected Pareto optimal fronts and disconnected Pareto optimal sets.

                    \item Many problems should be multimodal, and a few deceptive problems should also be covered.
                    \item The majority of test problems should be non-separable.
                    \item Both non-separable and multimodal problems should also be addressed.
                \end{enumerate}
            Ref[ “A Review of Multi-Objective Test Problems and a Scalable Test Problem Toolkit”, 2006]

            \section{Problem Suite: CEC 2009}
            Competition on “Performance Assessment of Constrained / Bound Constrained Multi-Objective Optimization Algorithms”. All problems are continuous, multi objective problems.

            \section{Physical. Real world problem}

                Computational models describing the behavior of complex physical systems are often used in 
                the engineering design field to identify better or optimal solutions with respect to previously 
                defined performance criteria. Multi-objective optimization problems arise and the set of optimal compromise 
                solutions (Pareto front) has to be identified by an effective and complete search procedure in order to let the 
                decision maker, the designer, to carry out the best choice.

                \subsection{Materials Selection in Mechanical Design}
                \subsection{Test generation}
                \subsection{Gold or oil search. Geodesy}
                \subsection{Space crafts}
                Problem 1: obtain a set of geometric design parameters to get minimum heat pipe mass and the maximum thermal conductance.
                Thus, a set of geometric design parameters lead to minimum pressure total cost and maximum pressure vessel volume. 
                The alternative solutions are very difficult to be adopted to practical engineering decision directly. 
                Ref[Multi-Objective Optimization Problems in Engineering Design Using Genetic Algorithm Case Solution]




    \chapter{Related work}


    \chapter{Conclusion}
        \lipsum[1]

    % must be invoked for correct page numbering in the appendix and all lists
    \backmatter
    
    \appendix
    \chapter{Appendix}
    \section{Additional Information}
    \lipsum[1]
    
    \section{More Important Information}
    \lipsum[1]
  
\end{document}